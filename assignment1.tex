\documentclass[12pt, a4paper, oneside]{article}
\usepackage{amsmath, amsthm, amssymb, bm, graphicx, hyperref, mathrsfs,tabularx}
\newenvironment{problem}{\stepcounter{problemname}\par\noindent\textsc{Problem \arabic{problemname}. }}{\\\par}
\newenvironment{solution}{\par\noindent\textsc{Solution. }}{\\\par}
\newenvironment{note}{\par\noindent\textsc{Note of Problem \arabic{problemname}. }}{\\\par}


\title{\textbf{Homework}}
\author{Xunyi Jiang 12010938}
\date{\today}
\linespread{1.5}
\newcounter{problemname}

\begin{document}

\maketitle

\begin{problem}
(5 points) Let p, q be the propositions\\
p: You get 100 marks on the final.\\
q: You get an A in this course.\\
Write these propositions using p and q and logical connectives (including negations).\\
(a) You do not get 100 marks on the final.\\
(b) You get 100 marks on the final, but you do not get an A in this course.\\
(c) You will get an A in this course if you get 100 marks on the final.\\
(d) If you do not get 100 marks on the final, then you will not get an A in this course.\\
(e) Getting 100 marks on the final is sufficient for getting an A in this course.\\
(f) You get an A in this course, but you do not get 100 marks on the final.\\
(g) Whenever you get an A in this course, you got 100 marks on the final.\\
\end{problem}
\begin{solution}
    (a) $\lnot p$\\
    (b) $p \land \lnot q$\\
    (c) $p \rightarrow q$\\
    (d) $\lnot p \rightarrow \lnot q$\\
    (e) $p \rightarrow q$\\
    (f) $q \land \lnot p$\\
    (g) $q \rightarrow p$\\
\end{solution}


\begin{problem}
(5 points) Construct a truth table for each of these compound propositions.\\
(a) $(p \oplus  q) \rightarrow (p \land q)$\\
(b) $(p \leftrightarrow  q) \oplus (\lnot p \leftrightarrow q)$\\
(c) $(p \oplus q) \rightarrow (p \oplus \lnot q)$\\
\end{problem}
\begin{solution}
    (a)\quad
    \begin{tabular}{|c|c|c|c|c|}
        \hline
        $p$ & $q$ & $p \oplus  q$ & $p \land q$ & $(p \oplus  q) \rightarrow (p \land q)$\\\hline
        T & T & F & T & T\\\hline
        T & F & T & F & F\\\hline
        F & T & T & F & F\\\hline
        F & F & F & F & T\\\hline
    \end{tabular}\\[12pt]
    (b)\quad
    \begin{tabular}{|c|c|c|c|c|}\hline
        $p$ & $q$ & $p \leftrightarrow q$ & $\lnot p \leftrightarrow q$ & $(p \leftrightarrow  q) \oplus (\lnot p \leftrightarrow q)$\\\hline
        T & T & T & F & T\\\hline
        T & F & F & T & T\\\hline
        F & T & F & T & T\\\hline
        F & F & T & F & T\\\hline        
    \end{tabular}\\[12pt]
    (c)\quad
    \begin{tabular}{|c|c|c|c|c|}\hline
        $p$ & $q$ & $p \oplus q$ & $p \oplus \lnot q$ & $(p \oplus q) \rightarrow (p \oplus \lnot q)$\\\hline
        T & T & F & T & T\\\hline
        T & F & T & F & F\\\hline
        F & T & T & F & F\\\hline
        F & F & F & T & T\\\hline         
    \end{tabular}
\end{solution}


\begin{problem}
(5 points) Use truth tables to decide whether or not the following two propositions are equivalent.\\
(a) $p \rightarrow q$ and $\lnot p \vee  q $ (This is the Useful Law.)\\
(b) $p \oplus q $ and $\lnot p \vee \lnot q$\\
(c) $(p \rightarrow q) \rightarrow r $ and $p \rightarrow (q \rightarrow r)$\\
(d) $(\lnot q  \land  (p \rightarrow q))$ and $\lnot p$\\
(e) $(p \vee q) \rightarrow r$ and $(p \rightarrow r) \land (q \rightarrow r)$\\
\end{problem}
\begin{solution}
    (a)\quad
    \begin{tabular}{|c|c|c|c|c|}\hline
        $p$ & $q$ & $p \rightarrow q$ & $\lnot p \vee q$ & $(p \rightarrow q) \leftrightarrow (\lnot p \vee q)$ \\\hline
        T & T & T & T & T\\\hline
        T & F & F & F & T\\\hline
        F & T & T & T & T\\\hline
        F & F & T & T & T\\\hline    
    \end{tabular}\\[12pt]
    \qquad $p \rightarrow q$ is equivalent to $\lnot p \vee q$.\\
    (b)\quad
    \begin{tabular}{|c|c|c|c|c|}\hline
        $p$ & $q$ & $p \oplus q$ & $\lnot p \vee \lnot q$ & $(p \oplus q) \leftrightarrow (\lnot p \vee \lnot q)$\\\hline
        T & T & F & F & T \\\hline
        T & F & T & T & T \\\hline
        F & T & T & T & T \\\hline
        F & F & F & T & F \\\hline    
    \end{tabular}\\[12pt]
    \quad$p \oplus q$ is not equivalent to $\lnot p \vee \lnot q$.\\[12pt]
    (c)\quad
    \begin{table}[h]
        \begin{tabular}[H]{|c|c|c|c|c|c|c|c|}\hline
        $p$ & $q$ & $r$ & $p \rightarrow q$ & $q \rightarrow r$ & $(p \rightarrow q) \rightarrow r$ & $p \rightarrow (q \rightarrow r)$ & \begin{tabular}[c]{@{}l@{}}
            $((p \rightarrow q) \rightarrow r) \Leftrightarrow$ \\ $(p \rightarrow (q \rightarrow r))$
        \end{tabular}\\\hline
        T & T & T & T & T & T & T & T\\\hline
        T & T & F & T & F & F & F & T\\\hline
        T & F & T & F & T & T & T & T\\\hline
        F & T & T & T & T & T & T & T\\\hline
        T & F & F & F & T & T & T & T\\\hline
        F & T & F & T & F & F & T & F\\\hline
        F & F & T & T & T & T & T & T\\\hline
        F & F & F & T & T & F & T & F\\\hline        
        \end{tabular}
    \end{table}\\[12pt]
    $(p \rightarrow q) \rightarrow r$ is not equivalent to $p \rightarrow (q \rightarrow r)$.\\[12pt]
    (d)\quad
    \begin{tabular}{|c|c|c|c|c|}\hline
        $p$ & $q$ & $p \rightarrow q$ & $\lnot q \land (p \rightarrow q)$ & $ (\lnot q \land (p \rightarrow q)) \leftrightarrow \lnot p$\\\hline
        T & T & T & F & F \\\hline
        T & F & F & F & T \\\hline
        F & T & T & F & F \\\hline
        F & F & T & T & T \\\hline        
    \end{tabular}\\[12pt]
    \qquad $\lnot q \land (p \rightarrow q)$ is not equivalent to $\lnot p$.\\[12pt]
    (e)\quad
    \begin{tabular}{|c|c|c|c|c|c|c|c|c|}\\\hline
        $p$ & $q$ & $r$ & $p \vee q$ & $(p \vee q) \rightarrow r$ & $p \rightarrow r$ & $q \rightarrow r$ & \begin{tabular}[c]{@{}l@{}}
            $(p \rightarrow r) \land$ \\ $ (q \rightarrow r)$
        \end{tabular}  & 
        \begin{tabular}[c]{@{}l@{}}
            $((p \vee q) \rightarrow r) \leftrightarrow$ \\ $((p \rightarrow r) \vee (q \rightarrow r))$
        \end{tabular}\\\hline
        T & T & T & T & T & T & T & T & T\\\hline
        T & T & F & T &F& F& F& F &T\\\hline
        T & F & T & T &T& T& T& T &T\\\hline
        F & T & T & T &T& T& T& T &T\\\hline
        T & F & F & T &F& F& T& F &T\\\hline
        F & T & F & T &F& T& F& F &T\\\hline
        F & F & T & F &T& T& T& T &T\\\hline
        F & F & F & F &T &T &T& T &T\\\hline        
    \end{tabular}
    \quad $((p \vee q) \rightarrow r) \leftrightarrow$ is equivalent to $((p \rightarrow r) \vee (q \rightarrow r))$.\\[12pt] 
\end{solution}


\begin{problem}
(10 points) Use logical equivalences to prove the following statements.\\
(a) $\lnot (p \oplus q)$ and $p \leftrightarrow q$ are equivalent.\\
(b) $\lnot(p \rightarrow q) \rightarrow \lnot q$ is a tautology.\\
(c) $(p \rightarrow q) \rightarrow ((r \rightarrow p) \rightarrow (r \rightarrow q))$ is a tautology
\end{problem}
\begin{solution}
    (a) $\lnot (p \oplus q) \equiv \lnot ((\lnot p \land q) 
    \vee (p \land \lnot q)) \equiv (p \vee \lnot q) \land (\lnot p \vee q)$\\
    \quad $(p \rightarrow q) \land (q \rightarrow p) \equiv (p \vee \lnot q) \land (\lnot p \vee q)$\\
    $\therefore \lnot (p \oplus q) \equiv (p \rightarrow q) \land (q \rightarrow p)$.\\
    (b) $\lnot(p \rightarrow q) \rightarrow \lnot q \equiv \lnot (p \land \lnot q) \vee \lnot q \equiv \lnot p \vee q \vee \lnot q = \lnot p \vee \text{T} \equiv $T.\\
    (c) $(p \rightarrow q) \rightarrow \lnot (\lnot r \vee p) \vee (\lnot r \vee q) \equiv (p \land \lnot q) \vee ((r \land \lnot p) \vee (\lnot r \vee q)) \equiv $\\
    $(p \land \lnot q) \vee (r \vee \lnot r) \land (\lnot p \vee r) \equiv (p \land \lnot q) \vee ((\text{T} \vee q) \land (\lnot p \vee \lnot r) \vee q) \equiv$\\
    $((p \land \lnot q) \vee T ) \land (p \land \lnot q) \vee \lnot p \vee \lnot r \vee q \equiv T \land ( T \vee T \vee \lnot r \equiv )$T.
\end{solution}


\begin{problem}
    (5 points) Explain, without using a truth table, why $(p \vee \lnot q) \land (q \vee \lnot r) \land (r \vee \lnot p)$ is true, when p, q, and r have the same truth value and it is false otherwise.\\
\end{problem}
\begin{solution}
    When p,q,r have the same truth value.\\
    %$(p \land (q \vee \lnot r) \land (r \vee \lnot p)) \vee (\lnot q \land (q \vee \lnot r) \land (r \vee \lnot p)) \equiv  $
    $(p \vee \lnot q) \land (q \vee \lnot r) \land (r \vee \lnot p) \equiv (q \rightarrow p) \land (r \rightarrow q) \land (p \rightarrow r)$\\
    Then there are two different cases.\\
    \textbf{Case 1:} p is T, since $(q \rightarrow p), (p \rightarrow r), (r \rightarrow q) \text{ all need to be true}$\\
    $\Rightarrow r $ needs to be true.$\Rightarrow q$ needs to be true. $\Rightarrow$ same truth value.\\
    \textbf{Case 2:} p is F, then $q$ needs to be false, then r needs to be false.  
\end{solution}


\begin{problem}
    (10 points) Determine whether or not the following two are logically equivalent, and explain your answer.\\
    (a) $(p \rightarrow q) \vee (p \rightarrow r) \text{ and } p \rightarrow (q \vee r)$\\
    (b) $\lnot p \rightarrow (q \rightarrow r) \text{ and } q \rightarrow (p \vee r)$\\
    (c) $(p \rightarrow q) \land (p \rightarrow r) \text{ and } p \rightarrow (q \land r).$\\
    (d) $(p \vee q) \rightarrow r \text{ and } (p \rightarrow r) \vee (q \rightarrow r)$.\\    
\end{problem}
\begin{solution}
    (a) $(p \rightarrow q) \vee (p \rightarrow r) \equiv (\lnot p \vee q) \vee (\lnot p \vee r) \equiv (\lnot p \vee q \vee r)$\\
    $p \rightarrow (q \vee r) \equiv \lnot p \vee q \vee r $\\ So these propersitions are equivalent.\\
    (b) $\text{left} = \lnot p \rightarrow (q \rightarrow r) = p \vee (\lnot q \vee r )$\\
    right = $\lnot q \vee (p \vee r)$\\
    So they are equivalent.\\
    (c) left = $(\lnot p \vee q ) \land (\lnot p \vee r) = \lnot p \vee (q \land r) = p \rightarrow (p \land r) $ = right\\
    They are equivalent.\\
    (d) left = $\lnot (p \vee q) \vee r  = (\lnot p \land \lnot q) \vee r = (\lnot p \vee r) \land (\lnot q \vee r) = (p \rightarrow r) \land (q \rightarrow r) $\\
    So they are not equivalent.
\end{solution}


\begin{problem}
(5 points) Let C(x) be the statement “x has a cat”, let D(x) be the statement “x has a dog”
and let F(x) be the statement “x has a ferret.” Express each of these sentences in terms of C(x),
D(x), F(x), quantifiers, and logical connectives. Let the domain consist of all students in your
class.\\
(a) A student in your class has a cat, a dog, and a ferret.\\
(b) All students in your class have a cat, a dog, or a ferret.\\
(c) Some student in your class has a cat and a ferret, but not a dog.\\
(d) No student in your class has a cat, a dog, and a ferret.\\
(e) For each of the three animals, cats, dogs, and ferrets, there is a student in your class who
has this animal as a pet.\\
\end{problem}
\begin{solution}
    (a) $\exists \text{x (C(x)} \land \text{D(x)} \land \text{F(x))}$\\
    (b) $\forall \text{x (C(x)} \vee \text{D(x)} \vee \text{F(x))}$\\
    (c) $\exists \text{x (C(x)} \land \text{F(x)} \land \lnot \text{D(x)}$.\\
    (d) $\forall \text{x} \lnot (C(x) \vee D(x) \vee F(x))$.\\
    (e) $\exists x_{1} C(x_{1}) \land \exists x_{2} D(x_{2}) \land \exists x_{3} F(x_{3})$
\end{solution}


\begin{problem}
(5 points) Let F(x, y) be the statement “x can fool y”, where the domain consists of all people in the world. Use quantifiers to express each of these statement.\\
(a) Everybody can fool Fred.\\
(b) Evelyn can fool everybody.\\
(c) There is no one who can fool everybody.\\
(d) Nancy can fool exactly two people.\\
(e) There is exactly one person whom everybody can fool.\\
(f) There is someone who can fool exactly one person besides himself or herself.\\
\end{problem}
\begin{solution}
    (a) $\forall x F(x,Fred)$.\\
    (b) $\forall y F(Evelyn , y)$.\\
    (c) $\forall x, \forall y \lnot F(x,y)$.\\
    (d) $\exists! y_{1}, y_{2}, F(Nancy, y_{1}) \land F(Nancy, y_{2})$.\\
    (e) $\exists! y, \forall x, F(x,y)$.\\
    (f) $\exists x, \exists y, y \neq x, F(x,y)$.\\ 
\end{solution}


\begin{problem}
(5 points) Express the negations of each of these statements so that all negation symbols
immediately precede predicates.\\
(a) $\forall y \exists x \exists z (T(x, y, z) \vee Q(x, y))$.\\
(b) $\exists x\exists y P(x, y) \land \forall x\forall y Q(x, y)$\\
(c) $\exists x\exists y(Q(x, y) \leftrightarrow Q(y, x))$\\
(d) $\forall x\exists y\forall z T(x, y, z)$\\
\end{problem}
\begin{solution}
    (a) $\exists y \forall x \forall z (\lnot T(x, y, z) \land \lnot Q(x,y))$\\
    (b) $\forall x \forall y \lnot P(x,y) \vee \exists x \exists y \lnot Q(x,y)$\\
    (c) $\forall x \forall y (\lnot Q(x, y) \oplus \lnot Q(y, x))$\\
    (d) $\exists x \forall \exists z \lnot T(x,y,x)$\\
\end{solution}


\begin{problem}
(5 points) Prove that if $p \land q, p \rightarrow \lnot(q \land r), s \rightarrow r, then \lnot s$.
\end{problem}
\begin{solution}
    \begin{tabular}{l l}
        \textbf{Step} & \textbf{Reason}\\
        1.$p \land q$ & Premise\\
        2.p & Simplification using (1)\\
        3.$p \rightarrow \lnot(q \land r)$ & Premise\\
        4.$\lnot(q \land r)$ & Modus ponens using (2) and (3)\\
        5.$p \rightarrow \lnot r$ & Simplification using (4)\\
        6.$\lnot r$ & Modus ponens using (2),(4)\\
        7.$s \rightarrow r$ & Premise\\
        8.$\lnot s$ & Modus tollens using (6),(7)\\
    \end{tabular}  
\end{solution}


\begin{problem}
(10 points) For each of these arguments, explain which rules of inference are used for each step.\\
    (a) “Somebody in this class enjoys whale watching. Every person who enjoys whale watching cares about ocean pollution. Therefore, there is a person in this class who cares about ocean pollution.”\\
    (b) “Each of the 93 students in this class owns a personal computer. Everyone who owns a personal computer can use a word processing program. Therefore, Zeke, a student in this class, can use a word processing program.”\\
    (c) “Each of five roommates, A, B, C, D, and E, has taken a course in discrete mathematics. Every student who has taken a course in discrete mathematics can take a course in algorithms. Therefore, all five roommates can take a course in algorithms next year.”\\    
\end{problem}
\begin{solution}
    (a) $p_{1}(x)$: x enjoys whale watching.\\
    $p_{2}(x)$: x cares about ocean pollution.\\
    Based on the discription above, $\exists x p_{1}(x), p_{1}(x) \rightarrow p_{2}(x)$\\ Using Modus ponens, we can get $\exists x, p_{2}(x)$.\\
    (b) $p_{1}(x)$: x owns a personal computer.\\
    $p_{2}(x)$: x can use a word processing program.\\
    Based on the discription above, $\forall xp_{1}(x),p_{1} \rightarrow p_{2}(x)$\\
    Using Modus ponens, we can get $\forall x, p_{2}(x)$\\ Since Zeke is in this class, then we know that $p_{2}(Zeke)$ is true.\\  
    (c) $p_{1}(x)$: x has taken a course in discrete mathematics.\\
    $p_{2}(x)$: x can take a course in algorithms.\\
    domain: x in \{A,B,C,D,E\} are the five roommates.\\
    $\forall x p_{1}(x) \rightarrow p_{2}(x)$ Using Modus ponens, we can get $\forall x p_{2}(x)$, which means all five roommates cna take a course in algorithms next year.  
\end{solution}


\begin{problem}
    (5 points)
    (a) Give the negation of the statement
    $\forall n \in \mathbb{N}  (n^3 + 6n + 5) is odd \Rightarrow n is even).$
    (b) Either the original statement in (a) or its negation is true. Which one is it and explain why?    
\end{problem}
\begin{solution}
    (a) $\exists n \in \mathbb{N} (n^3+6n+5) is odd \Rightarrow n is odd$\\
    (b) Proof by contradiction :\\
    \begin{proof}
         Suppose n is odd, then let $n = 2k+1, k \in \mathbb{N}$\\
    $n^3 + 6n + 5 = (2k+1)^3 + 6(2k+1) + 5 = 2(4k^3+2k^2+7k+6)$ is even,\\ which is contradict to $(n^3 + 6n + 5)$ is odd. 
    \end{proof}
       So the original statement in (a) is true.
\end{solution}


\begin{problem}
     (5 points) Give a direct proof that: Let a and b be integers. If $a^2 + b^2$ is even, then a + b is even.
\end{problem}
\begin{solution}
    \begin{proof}
        Let $a^2+ b^2$ is even, then there are two situations.\\
        1.$a^2$ is even and $b^2$ is odd\\
        then a is even and b is odd, $a+b$ is even.\\
        2.$a^2$ is odd and $b^2$ is even\\
        then a is odd and b is even, $a+b$ is even.
    \end{proof}
    So if $a^2 + b^2$ is even, then a + b is even.
\end{solution}



\begin{problem}
    Prove or disprove that if a and b are rational numbers, then $a^b$ is also rational.    
\end{problem}
\begin{solution}
    \begin{proof}
        We can find a counter-example, let a = 2, b = 1/2, \\then $a^b = \sqrt{2}$ is not a rational number.  
    \end{proof}
    This statement is wrong.
\end{solution}


\begin{problem}
    Prove that $2^{1/3}$ is irrational.
\end{problem}
\begin{solution}
    \begin{proof}
        Argue by contradiction, assum $2^{1/3}$ is a rational number, \\$2^{1/3} = p/q$,which p, q are relatively prime number.\\
        $p^3 = 2q^3$ then $p^3$ is even, so p is even, let $p = 2k, k \in \mathbb{N}$\\
        $q^3 = 4k^3$,which means $q^3$ is even, so q is even. so p and q is not relatively prime number, which is contradict to the assumption  
    \end{proof}.\\
\end{solution}



\begin{problem}
    Suppose that we have a theorem: “$\sqrt n$ is irrational whenever n is a positive integer that is not a perfect square.” Use this theorem to prove that $\sqrt 2 + \sqrt 3$ is irrational.
\end{problem}
\begin{solution}
    \begin{proof}
        Argue by contradiction, assum $\sqrt 2 + \sqrt 3$ is rational, let  $\sqrt 2 + \sqrt 3 = \frac{p}{q}$, where p and q are relatively prime integers.\\
        Then $(\sqrt 2 + \sqrt 3)^2 = p^2 \rightarrow 5q^2 + 2\sqrt 6 q = p^2$\\
        Then we can get $\sqrt 6 = \frac{p^2-5q^2}{2q}$, so $\sqrt 6 $ is a rational number. Contradiction
    \end{proof}
    But it contradicts to the theorem that is mentioned in the problem.
\end{solution}



\begin{problem}
(5 points) Please read the following description carefully and answer questions. We consider a single object sealed-bid second-price auction (to be explained). In this auction, an auctioneer is responsible for selling a product, and bidders bid for the product. The winner of the auction wins
the product and pays for it. The detailed settings are as follows:
\begin{itemize}
\item There is one product to be sold.
\item There are N bidders, denoted by N = {1, 2, ..., N}. Bidder $n \in \mathbb{N}$ has a valuation over the product of $v_{n}$.
\item Every bidder submits his or her bid in a sealed envelope, so other bidders do not know his
or her bid. Bidder $n \in \mathbb{N}$ submits a bid of $b_{n}$.
\item  After receiving the bids from all bidders, the auctioneer announces the winner and payment.
The winner is the bidder who submits the highest bid. The payment of this winner is the
second highest bid. For example, consider three bidders. Suppose $b_{1} = 2, b_{2} = 4, b_{3} = 5$.
Then, the winner is bidder 3, and the payment is the second highest bid 4.
\item If multiple bidders have the same bid, then they draw a lottery. Each of them has equally
probability of winning. In this case, the payment is equal to their bids. For example,
consider three bidders. Suppose $b_{1} = 2, b_{2}= 5, b_{3} = 5$. The winner is either 2 or 3 with
equal probability. The payment is 5.
\item After the auction, the payoffs of the bidders are as follows:
    \begin{itemize}
        \item If bidder n loses, his or her payoff is zero.
        \item  If bidder n wins, his or her payoff is equal to its valuation $v_{n}$ minus the payment.
    \end{itemize}
    
For bidder n, the higher payoff, the better.
\end{itemize}
Now, suppose you are a bidder in this auction, e.g., bidder n, and you do not know any other
bidders' valuations and bids. You know your valuation $v_{n}$. You can choose your bid $b_{n}$ to maximize your payoff. Prove that for an arbitrary bidder $n \in \mathbb{N}$ , submitting a bid $b_{n} = v_{n}$ will always lead
to a payoff that is no smaller than submitting a bid with $b_{n} \neq v_{n}$.
(Note: This second-price auction is commonly used, due to the property that bidders are willing
to submit their valuation as their bid. )
(Hint: Use proof by cases; consider the highest bid of the others, and compare it with your
valuation $v_{n}$; enumerate all possibilities.)
\end{problem}
\begin{solution}
    Since we do not need to consider the loss situation, so if we win, we can let $b_{n} = max{b_{1},...b_{2}}$ 
    Suppose $k = max{b_{1},...,b_{n}}/b_{n}$, then we can enumerate all the situations in the table below.\\
    \begin{tabular}{|c|c|c|c|}\hline
        & $v_{n} = b_{n}$ & $b_{n} < v_{n}$ & $b_{n} > v_{n}$\\\hline
        $k < v_{n}$ & $v_{n} - k$ &  $v_{n} - k$ &  $v_{n} - k$\\\hline
        $k = v_{n}$ & 0 & 0 & 0\\\hline
        $k > v_{n}$ & 0 & 0 &  
            \begin{tabular}{c c}
                $v_{n} - k < 0$, & $b_{n} > k$ \\ 
                0,& $b_{n} = k $\\
                0,& $b_{n} < k$\\                 
            \end{tabular}\\\hline
        \end{tabular}\\[12pt]
    We can see that if we always let $b_{n} = v_{n}$, the payoff is no smaller than the other situations.
\end{solution}





% \begin{note}
%     Write note here. 
% \end{note}

\end{document}